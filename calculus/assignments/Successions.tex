\documentclass{article}

\usepackage{amsmath}
\usepackage{pgfplots}  
\usepackage{titling} 
\usepackage[utf8x]{inputenc}
\usepackage{lmodern,textcomp} 
\usepackage[super]{nth}

\pgfplotsset{compat=newest}

\title{Actividad en Clase \\ Succesiones y Progresiones}
\author{Nathan Alspaugh y Gabriela Cortes \\ Colegio Real Royal School}

\begin{document}
\maketitle
\newtheorem{successions}{Problema}
\begin{successions}
   Determina el \nth{7} termino de 200, 100, 50\dots
   \begin{equation}
         \begin{aligned}
              & a_{n} = a_{1} * r{n-1} \\
                & a_{7} = 200 * \bigg(\frac{1}{2}\bigg)^{6} \\
                & a_{7} = 200 * \dfrac{1}{2^6} \\
                & a_{7} = \frac{200}{64} \\
                & a_{7} = \frac{25}{8} \\
         \end{aligned}
   \end{equation}
\end{successions}
\begin{successions}
    Determina el razon si el \nth{1} termino es \(\frac{3}{5}\) y el \nth{5} termino es \(\frac{1}{135}\)
    \begin{equation}
        \begin{aligned}
            & r = \sqrt[n-1]{\frac{a_{n}}{a_{1}}} \\
            & r = \sqrt[4]{\frac{\frac{1}{135}}{\frac{3}{8}}} \\
            & r = \sqrt[4]{\frac{8}{405}} \\
            & r = \sqrt[4]{\frac{1}{81}} \\
            & r = \frac{1}{3} \\
        \end{aligned}
    \end{equation}
\end{successions}
\begin{successions}
    Determina el numero de terminos de -2, -6, ..., -162
    \begin{equation}
        \begin{aligned}
            & n = \frac{\log{n} - \log{a_{1}} + \log{r}}{\log{r}}; && r = \frac{a_{n}}{a_{n-1}} \\
            & n = \frac{\log{-162} - \log{-2} + \log{3}}{\log{3}} && r = \frac{-6}{-2} = 3 \\
            & n = \frac{\log{-162} - \log{-2} + \log{3}}{\log{3}} \\
            & n = 5
        \end{aligned}
    \end{equation}
\end{successions}
\begin{successions}
    Encuentra la suma de los primeros 9 terminos de -5, 10, -20\dots
    \begin{equation}
        \begin{aligned}
            & S_{n} = \frac{a_{1}(1 - r^{n})}{1 - r} && r = \frac{a_{n}}{a_{n-1}}  \\
            & S_{9} = \frac{-5(1 - (-2)^{9})}{1 - (-2)} && r = \frac{10}{-5} = -2 \\
            & S_{9} = \frac{-5(1 + 512)}{1 + 2} \\
            & S_{9} = \frac{-5(513)}{3} \\
            & S_{9} = \frac{-2565}{3} \\
            & S_{9} = -855
        \end{aligned}
    \end{equation}
\end{successions}

\end{document}