\section{Carbon}
Notes about carbon and organic chemistry!
\subsection{What is it?}
Carbon is an element which is characterized for being the 4th most abundant element in the universe. It is considered a fundamental pillas as it can form 4 covalent bonds which allows it to combine in various forms.

\,\\
Substances that have Carbon:
\begin{itemize}
    \item Hidrocarbons (Ethane)
    \item Carbohydrates (Glucose)
    \item Proteins (Amino Acids)
    \item Lipids (Phospholipids)
    \item Nucleic Acids
\end{itemize}
\subsection{Alcalines with Carbon}
There are 2 types of nomenclature for alcalines with carbon:
\begin{itemize}
    \item Lineal: The carbon atoms are in a line
    \item Ramified: The carbon atoms are in a branched structure
\end{itemize}

\,\\
The naming for alcalines with carbon is done by the number of carbon atoms in the chain. The prefixes are:
\begin{itemize}
    \item Meth-: 1 carbon atom
    \item Eth-: 2 carbon atoms
    \item Prop-: 3 carbon atoms
    \item But-: 4 carbon atoms
    \item Pent-: 5 carbon atoms
    \item Hex-: 6 carbon atoms
    \item Hept-: 7 carbon atoms
    \item Oct-: 8 carbon atoms
    \item Non-: 9 carbon atoms
    \item Dec-: 10 carbon atoms
    \item Undec-: 11 carbon atoms
    \item Dodec-: 12 carbon atoms
    \item Tridec-: 13 carbon atoms
    \item Tetradec-: 14 carbon atoms
    \item Pentadec-: 15 carbon atoms
    \item Hexadec-: 16 carbon atoms
\end{itemize}

\,\\
The other prefixes for general number naming are:
\begin{itemize}
    \item Di-: 2 atoms
    \item Tri-: 3 atoms
    \item Tetra-: 4 atoms
    \item Penta-: 5 atoms
    \item Hexa-: 6 atoms
    \item Hepta-: 7 atoms
    \item Octa-: 8 atoms
    \item Nona-: 9 atoms
    \item Deca-: 10 atoms
    \item Undeca-: 11 atoms
    \item Dodeca-: 12 atoms
    \item Trideca-: 13 atoms
    \item Tetradeca-: 14 atoms
    \item Pentadeca-: 15 atoms
    \item Hexadeca-: 16 atoms
\end{itemize}
\subsubsection{Lineal}
The lineal alcalines are the simplest form of alcalines with carbon. They are named using the prefix corresponding to the number of atoms and then -ane. They are characterized for having the carbon atoms in a line.

\,\\
Examples:
\begin{itemize}
    \item Methane: \chemfig{\ce{CH4}}
    \item Ethane: \chemfig{\ce{CH3}-\ce{CH3}}
    \item Propane: \chemfig{\ce{CH3}-\ce{CH2}-\ce{CH3}}
    \item Butane: \chemfig{\ce{CH3}-\ce{CH2}-\ce{CH2}-\ce{CH3}}
\end{itemize}
\subsubsection{Ramified}
The ramified alcalines are a more complex form of alcalines with carbon. The naming is more complicated and it is done by the number of atoms in the longest chain and then the number of atoms in the branches. The branches are named by the number of atoms and then the position in the chain. For example, 2-Methylbutane has 4 carbon atoms in the longest chain and 1 carbon atom in the branch in the 2nd position.

\begin{center}
    \chemfig{\ce{CH3}-\ce{CH2}-\ce{CH2}-\ce{CH}(-[6]\ce{CH3})-\ce{CH3}}
\label{fig:2-Methylbutane}
\end{center}

\subsection{Isomeros}
Isomeros son un tipo de radical que se forma por la reorganización de los átomos en una molécula. Los isómeros son moléculas que tienen la misma fórmula molecular, pero diferente estructura. Los isómeros se pueden clasificar en dos tipos: isómeros estructurales y isómeros espaciales. Los isómeros estructurales son moléculas que tienen la misma fórmula molecular, pero diferente estructura.

\subsection*{Ejemplo con \ce{CH12}}

\begin{itemize}
    \item \chemfig{\ce{CH3}-\ce{CH2}-\ce{CH2}-\ce{CH2}-\ce{CH3}}, Pentano
    \item \chemfig{\ce{CH3}-\ce{CH2}-\ce{CH2}-\ce{CH}(-[6]\ce{CH3})-\ce{CH3}}, 2-Metilbutano
    \item \chemfig{\ce{C}(-[2]\ce{CH3})(-[4]\ce{CH3})(-[6]\ce{CH3})(-[8]\ce{CH3})}, 2,2-Dimetilpropano
    \end{itemize}
    \qquad
    \subsection*{Ejemplo con \ce{C23H48}}
    \chemfig{\ce{CH3}-\ce{CH}(-[6]\ce{CH3})-\ce{CH2}-C(-[2]\ce{CH3})(-[6]\ce{CH3})-\ce{CH2}-C(-[6]\ce{CH2}(-[6]\ce{CH3}))-\ce{CH2}-\ce{CH2}-\ce{CH}(-[2]\ce{CH2}(-[2]\ce{CH3}))-\ce{CH2}-\ce{CH2}(-[6]\ce{CH2}(-[6]\ce{CH2}(-[6]\ce{CH2}(-[6]\ce{CH3}))))-\ce{CH3}}, 6,9-Dietil-2,4,4,11-Tetrametilpentadecano
    \qquad
\subsection*{Ejemplo con diagrama ramificada}
\chemfig{.-[1].-[7].(-[6].)-[1].-[7].(-[6].)-[1].-[7].(-[6].(-[7].))-[1].-[7].}
3-Ethyl-5,7-DimethylNonane (3-Etil-5,7-Dimetilnonano)
\quad
Notes: Start numbering from the side closest to the first branch, if the 2 1st branches are at the same distance, try the second farthest branch. If all the branches are the same distance form the ends, try alphabetically. In this example you would start from the right and then go to the left because the 1st and 2nd distanced branches are the sam for both sides and then you start alphabetically which would be Ethyl from the right.

