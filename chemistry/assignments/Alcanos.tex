\documentclass{article}

\usepackage{mhchem}
\usepackage{chemfig}

\title{Taller \\ Nomenclatura de Alcanos}
\author{Nathan Alspaugh \\ Colegio Real Royal School}

\begin{document}

\maketitle
\newtheorem{namefromgraph}{Punto}
\begin{namefromgraph}
    Da el nombre a cada uno de los siguientes alcanos.
    \begin{equation}
        \chemname{\chemfig{.-[1].(-[2].)-[7].-[1].(-[2].)-[7].-[1].-[7].}}{2,4-Dimethylheptane (2,4-Dimetilheptano)}
    \end{equation}
    \begin{equation}
        \chemname{\chemfig{\ce{CH3}-[8]\ce{CH2}-[8]\ce{CH}(-[2]\ce{CH3})-[8]\ce{CH2}-[8]\ce{CH}(-[2]\ce{CH3})-[8]\ce{CH3}}}{2,4-Dimethylhexane (2,4-Dimetilhexano)}
    \end{equation}
    \begin{equation}
        \chemname{\chemfig{\ce{CH3}-[8]\ce{CH2}-[8]\ce{CH}(-[6]\ce{CH3}(-[8]\ce{CH3}))-[8]\ce{CH}(-[2]\ce{CH2}(-[8]\ce{CH3}))-[8]\ce{CH2}-[8]\ce{CH3}}}{3,4-Diethylhexane (3,4-Dietilhexano)}
    \end{equation}
    \begin{equation}
        \chemname{\chemfig{\ce{CH3}-[8]\ce{CH2}-[8]\ce{CH}(-[6]\ce{Cl})-[8]\ce{CH3}}}{2-Chlorobutane (2-Clorobutano)}
    \end{equation}
    \begin{equation}
        \chemname{\chemfig{.-[1].-[7].-[1].(-[2]\ce{Cl})-[7].-[1].(-[2].)-[7].-[1].}}{5-Chloro-3-Methyloctane (5-Cloro-3-Metiloctano)}
    \end{equation}

\end{namefromgraph}
\begin{namefromgraph}
    Dibuja la formula de cada alcano

    \begin{equation}
        \chemname{\chemfig{\ce{CH3}-[0]\ce{CH2}-[0]\ce{CH}(-[2]\ce{CH2}(-[0]\ce{CH3}))-[0]\ce{CH3}}}{3-Ethylheptane (3-Etilheptano)}
    \end{equation}
    \begin{equation}
        \chemname{\chemfig{\ce{CH3}-[0]\ce{C}(-[2]\ce{CH3})(-[6]\ce{CH3})-[0]\ce{CH}(-[2]\ce{CH3})-[0]\ce{CH2}-[0]\ce{CH}(-[2]\ce{CH3})-[0]\ce{CH3}}}{2,2,3,5-Tetramethylhexane (2,2,3,5-Tetrametilhexano)}
    \end{equation}
    \begin{equation}
        \chemname{\chemfig{\ce{CH3}-[0]\ce{C}(-[2]\ce{CH3})(-[6]\ce{CH3})-[0]\ce{CH2}-[0]\ce{CH}(-[2]\ce{CH2}(-[8]\ce{CH3}))-[0]\ce{CH2}-[0]\ce{CH2}-[0]\ce{CH2}-[0]\ce{CH3}}}{4-Ethyl-2,2-Dimethyloctane (4-Etil-2,2-Dimetiloctano)}
    \end{equation}
    \begin{equation}
        \chemname{\chemfig{\ce{CH}(-[2]\ce{Br})(-[6]\ce{Br})-[0]\ce{CH}(-[2]\ce{Br})(-[6]\ce{Br})-[0]\ce{CH2}-[0]\ce{CH3}}}{1,1,2,2-Tetrabromopropane (1,1,2,2-Tetrabromopropano)}
    \end{equation}
    \begin{equation}
        \chemname{\chemfig{\ce{CH3}-[0]\ce{CH}(-[2]\ce{Cl})(-[6]\ce{CH3})-[0]\ce{C}(-[2]\ce{Cl})-[0]\ce{CH3}}}{2,3-Dichloro-2-Methylbutane (2,3-Dicloro-2-Metilbutano)}
    \end{equation}

\end{namefromgraph}
\end{document}