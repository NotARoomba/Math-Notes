\documentclass{article}
\usepackage{fancyhdr}
\usepackage{mhchem}
\usepackage{chemfig}
\begin{document}
\pagestyle{fancy}
\fancyhead{} % clear all header fields
\fancyhead[LO,CE]{\textbf{Laboratorio producci\'on de acetileno}}
\fancyhead[RO,RO]{\thepage}
\begin{titlepage}
      \centering
      {\bfseries\LARGE Colegio Real Royal School \par}
      \vspace{1cm}
      {\scshape\Large Qu\'imica \par}
      \vspace{3cm}
      {\scshape\Huge Producci\'on de acetileno (elaboraci\'on de un soplete casero) \par}
      % \vspace{3cm}
      % {\itshape\Large Proyecto Fin de Carrera \par}
      \vfill
      % {\Large Autor: \par}
      {\Large Nathan Alspaugh, 11B  \par}
      % {\Large 11B \par}
      \vfill
      % {\Large Profesor: \par}
      {\Large Mr. Diogenes Visbal \par}
      \vfill
      {\Large Febrero 2025 \par}
\end{titlepage}
\setcounter{section}{1}
\section*{Introducci\'on}
\subsection{Objectivo}
Los alquenos son hidrocarburos insaturados que contienen un doble enlace carbono-carbono. En esta pr\'actica, se producir\'a acetileno, un alqueno, a partir de carburo de calcio y agua. El acetileno es un gas inflamable que se utiliza en la soldadura y corte de metales. Con el fin de comprobar la reactividad de los alquenos y su uso en la industria.
\subsection{Materiales}
\begin{itemize}
      \item Botella de pl\'astico
      \item Carburo de calcio
      \item Agua
      \item Caja de fosforos
      \item Set de infusión de macrogoteo
      \item Globo
      \item Tijeras
      \item Pistola de silicona
\end{itemize}
\subsection{Procedimiento}
\begin{enumerate}
      \item Crea un hueco en la tapa de la botella de pl\'astico.
      \item Con la pistola de silicona, pega el set de infusión de macrogoteo en hueco de la tapa.
      \item Llena la botella con 4-5 piedras de carburo de calcio.
      \item Prepara el globo y la jeriga.
      \item Llena la botella con agua y tapa r\'apidamente.
      \item Coloca el globo en la punta del set de infusión de macrogoteo y abre la llave de la jeriga.
      \item Cuando se llene el globo, quita el globo y remmplaza lo por la jeriga.
      \item Enciende un f\'osforo y ac\'ercalo a la punta de la jeriga.
\end{enumerate}
\section*{Analisis de Resultados}
\subsection{Observaciones}
La reacci\'on entre el carburo de calcio y el agua produce acetileno, un gas inflamable. Cuando hechamos agua en la botella, se produce una reacci\'on efervescente. El acetileno pasa por el set de infusión de macrogoteo y se sale por la jeriga. Al encender un f\'osforo y acercarlo a la punta de la jeriga, el acetileno se inflama. Este proceso mustra una llama azul en la base de la llama y una llama amarilla en la punta de la llama. La llama azul es evidencia de una reacci\'on de combusti\'on completa, mientras que la llama amarilla es evidencia de una reacci\'on de combusti\'on incompleta.
\subsection{Preguntas}
\begin{enumerate}
      \item ?`Qu\'e es el acetileno y cuales son sus usos?
            \\~\\
            El acetileno es un gas inflamable que se utiliza en la soldadura y corte de metales.
      \item Por medio de la ecuaci\'on qu\'imica balanceada, representa la reacci\'on que se llev\'o a cabo durante la pr\'actica del laboratorio.
            \\~\\
            \ce{CaC2 + 2H2O -> C2H2 + Ca(OH)2}
      \item Por medio de la ecuaci\'on qu\'imica balanceada, representa la combusti\'on completa y incompleta del acetileno.
            \\~\\
            \ce{2C2H2 + 5O2 -> 4CO2 + 2H2O} (combusti\'on completa)\\~\\
            \ce{2C2H2 + 3O2 -> 4CO + 2H2O} (combusti\'on incompleta partial)
            \\~\\
            \ce{2C2H2 + O2 -> 4C + 2H2O} (combusti\'on incompleta minima)
      \item ?`Qu\'e son los equipos de oxicarte o oxiaceti\-leno y que usos tiene a nivel industrial?
            \\~\\
            Los equipos de oxicarte o oxiacetileno son usados en la soldadura y corte de metales. Se utilizan para fundir y cortar metales.
      \item Elabora una conclusi\'on de lo aprendido durante la pr\'actica del laboratorio.
            \\~\\
            En esta pr\'actica, aprend\'i a producir acetileno a partir de carburo de calcio y agua. Tambi\'en aprend\'i sobre la reactividad de los alquenos y su uso en la industria.
\end{enumerate}

\end{document}