\documentclass[12pt]{article}
\usepackage[spanish]{babel}
\usepackage{geometry}
\usepackage{graphicx}
\usepackage{setspace}
\usepackage{hyperref}
\usepackage{apacite}
\usepackage{fancyhdr}
\usepackage{titlesec}
\usepackage{nopageno}
\usepackage{tocloft}


%  small letter size for title 12fnt
%     key words in the footer
% Justification lins
% only marco teorico and metodologia and anexos, encuesta/resultados
\geometry{a4paper, margin=1in}
\doublespacing
\title{\large{\textbf{AquaSolar: Sistema de Riego Sostenible para Huertas}}}
\author{Juan Andrés Del Toro R, Nathan Daniel Alspaugh y Juan Pablo Ortiz \\ Medio Ambiente, Colegio Real Royal School \\ Undecimo Grado \\ Lic. Corina Corpas Santofimio \\ 21 de Mayo de 2025}
\date{}
\begin{document}
\pagestyle{fancy}
\cfoot{}
\fancyhead{} % clear all header fields
\fancyhead[LO,CE]{\textbf{AquaSolar:  Sistema de Riego Sostenible para Huertas
    }}
\fancyhead[RO,RO]{\thepage}
\maketitle

\newpage
\renewcommand{\thesection}{}
\renewcommand{\thesubsection}{}
\renewcommand{\thesubsubsection}{}
\renewcommand{\cftsecleader}{\cftdotfill{\cftdotsep}}
\titleformat{\section} % Command to modify
{\normalfont\Large\bfseries} % Font style for the section title
{\thesection} % Section number (optional)
{0pt} % Space between number and title
{} % Code before the title (optional)

% Customize subsection formatting
\titleformat{\subsection} % Command to modify
{\normalfont\large\bfseries} % Font style for the subsection title
{\thesubsection} % Subsection number (optional)
{1em} % Space between number and title
{} % Code before the title (optional)

% Customize subsubsection formatting
\titleformat{\subsubsection} % Command to modify
{\normalfont\normalsize\bfseries} % Font style for the subsubsection title
{\thesubsubsection} % Subsubsection number (optional)
{2em} % Space between number and title
{} % Code before the title (optional)
\newpage
\section*{Resumen}
A lo largo de la historia, la humanidad ha enfrentado retos en la gestión de los recursos naturales, particularmente en el uso de fuentes de energía no renovables. Según la Agencia Internacional de Energía, la sobreexplotación de recursos como la electricidad, proveniente de fuentes fósiles, representa un desafío creciente para la sostenibilidad global. En este contexto, es urgente buscar soluciones sostenibles que promuevan el uso de energías limpias. El proyecto Aqua Solar responde a esta necesidad mediante la implementación de un sistema de riego automatizado que aprovecha la energía solar como fuente principal.

El objetivo central del proyecto es diseñar un sistema de riego eficiente, impulsado por energía solar, para huertos escolares, con el fin de reducir costos operativos, maximizar la utilización de energía, y fomentar la sostenibilidad ambiental. Al integrar paneles fotovoltaicos en el sistema, se garantiza la autosuficiencia energética, eliminando la dependencia de fuentes no renovables y promoviendo el uso de energía limpia. Además, el proyecto incluye un controlador automático que ajusta el riego según las necesidades específicas de las plantas, mejorando la gestión eficiente del agua.

Con este enfoque, el proyecto no solo busca optimizar el uso de la energía solar, sino también sensibilizar a los estudiantes sobre la importancia de las energías renovables y su impacto positivo en el medio ambiente.
\\~\\
\textbf{Palabras clave:} Energías limpias, energía solar, energías renovables, sostenibilidad, autosuficiencia energética
\newpage
\section*{Abstract}
Throughout history, humanity has faced challenges in managing natural resources, particularly in the use of non-renewable energy sources. According to the International Energy Agency, the overexploitation of resources such as electricity derived from fossil fuels represents a growing challenge for global sustainability. In this context, it is urgent to seek sustainable solutions that promote the use of clean energy. The Aqua Solar project addresses this need by implementing an automated irrigation system powered primarily by solar energy.

The core objective of the project is to design an efficient solar-powered irrigation system for school gardens to reduce operational costs, maximize energy utilization, and promote environmental sustainability. By integrating photovoltaic panels into the system, energy self-sufficiency is ensured, eliminating reliance on non-renewable sources and advancing the use of clean energy. Additionally, the project includes an automatic controller that adjusts irrigation based on the specific needs of the plants, improving efficient water management.

With this approach, the project not only aims to optimize the use of solar energy but also to raise awareness among students about the importance of renewable energy and its positive impact on the environment.
\\~\\
\textbf{Keywords:} Clean energy, solar energy, renewable energy, sustainability, energy self-sufficiency.
\newpage
\renewcommand{\contentsname}{Tabla de Contenido}
\tableofcontents
\newpage
\section{Antecedentes}
El desarrollo sostenible y la necesidad de reducir el impacto ambiental han cobrado cada vez más relevancia en la agenda educativa y científica, especialmente en contextos vulnerables al cambio climático como la región Caribe colombiana. En este sentido, el proyecto realizado por Salwa Bojanini, Sofía Llinás Hernández, Alejandro Pacheco Guerrero, Juan David Villarruel, Alejandro Mercado Pinedo y Santiago Torres Porras en el año 2023, en el Colegio Real de Barranquilla, representa un punto de partida fundamental para nuestro trabajo actual. Este proyecto planteó la optimización del mantenimiento de una huerta escolar mediante la implementación de un sistema de riego automatizado, impulsado por energía solar, como una forma concreta de avanzar hacia prácticas agrícolas sostenibles desde el ámbito educativo.

Dicho proyecto demostró la viabilidad técnica y ambiental del uso de paneles solares como fuente energética limpia, abordando no solo una problemática local —el consumo elevado de energía y recursos en actividades agrícolas escolares— sino también conectándose con objetivos globales como los planteados en la Agenda 2030 de la ONU. Al incorporar asesoría de expertos de la Universidad del Norte y adaptar el diseño del sistema a las condiciones específicas del entorno escolar, el proyecto logró desarrollar una solución autosostenible que ejemplifica cómo la educación puede ser motor de transformación ecológica.

A partir de esta base, nuestro proyecto de medio ambiente se articula con la visión propuesta por el trabajo anterior, reconociendo tanto su impacto como sus limitaciones, y busca ampliarlo en alcance y profundidad. Retomamos su enfoque en la energía fotovoltaica, su énfasis en la reducción de la huella de carbono y su intención pedagógica, para fortalecer la conciencia ambiental en nuestra comunidad educativa y avanzar hacia una cultura institucional verdaderamente comprometida con la sostenibilidad.
\newpage

\section{Planteamiento del problema}
\subsection{Tema}
Diseño e Implementación de un Sistema de Riego Sostenible para Huertas, Utilizando Energía Solar para Maximizar la Eficiencia en el Uso del Agua y Reducir el Impacto Ambiental.

\subsection{Pregunta Problema}
¿Cómo aprovechar de manera eficiente la energía solar para automatizar el riego en huertas, maximizando el uso del agua y reduciendo los costos operativos, mientras se minimiza el impacto ambiental?
\newpage
\section{Objetivos}
\subsection{Objetivo General}
Diseñar un sistema que aproveche la luz solar para automatizar el riego en huertos escolares, mejorando la eficiencia energética y promoviendo la sostenibilidad.


\subsection{Objetivos Específicos}
\begin{itemize}
    \item Desarrollar un controlador automático de riego que ajuste la cantidad y frecuencia del agua de acuerdo con las necesidades específicas de las plantas.
    \item Integrar paneles fotovoltaicos para alimentar el sistema de riego, disminuyendo la huella de carbono y la dependencia de energías no renovables.
    \item Fomentar la sostenibilidad ambiental y la educación sobre energías renovables, estableciendo un modelo replicable para otras huertas educativas.
\end{itemize}

\newpage
\section{Conclusión}
El acceso al agua potable es un desafío que ha acompañado a la humanidad a lo largo de la historia. En la actualidad, la crisis hídrica requiere soluciones innovadoras que permitan garantizar la disponibilidad y calidad del recurso de manera sostenible. En este sentido, Aqua Solar representa una propuesta tecnológica viable y eficaz, capaz de integrar energía solar en sistemas de abastecimiento de agua para comunidades con una infraestructura en un ecosistema de condiciones adversas.

Los hallazgos del proyecto demuestran que la energía fotovoltaica es una alternativa eficiente para la gestión del agua, reduciendo costos operativos y minimizando el impacto ambiental. A través de su implementación, se espera beneficiar a miles de personas, mejorando su calidad de vida y reduciendo los riesgos sanitarios asociados al consumo de agua no potable.

El impacto de Aqua Solar no se limita a sus beneficiarios directos, sino que sienta las bases para el desarrollo de nuevas estrategias de abastecimiento hídrico sostenible. A futuro, se plantea la posibilidad de expandir el proyecto a nivel regional, incorporando avances tecnológicos en almacenamiento de energía y automatización de los sistemas de distribución.

En definitiva, Aqua Solar demuestra que la combinación de tecnología y sostenibilidad es clave para enfrentar los retos actuales del acceso al agua. Su implementación no solo representa una solución innovadora, sino que también marca un precedente en la búsqueda de modelos sustentables que puedan replicarse a nivel global.

\end{document}